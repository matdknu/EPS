% Options for packages loaded elsewhere
\PassOptionsToPackage{unicode}{hyperref}
\PassOptionsToPackage{hyphens}{url}
%
\documentclass[
]{article}
\usepackage{amsmath,amssymb}
\usepackage{iftex}
\ifPDFTeX
  \usepackage[T1]{fontenc}
  \usepackage[utf8]{inputenc}
  \usepackage{textcomp} % provide euro and other symbols
\else % if luatex or xetex
  \usepackage{unicode-math} % this also loads fontspec
  \defaultfontfeatures{Scale=MatchLowercase}
  \defaultfontfeatures[\rmfamily]{Ligatures=TeX,Scale=1}
\fi
\usepackage{lmodern}
\ifPDFTeX\else
  % xetex/luatex font selection
\fi
% Use upquote if available, for straight quotes in verbatim environments
\IfFileExists{upquote.sty}{\usepackage{upquote}}{}
\IfFileExists{microtype.sty}{% use microtype if available
  \usepackage[]{microtype}
  \UseMicrotypeSet[protrusion]{basicmath} % disable protrusion for tt fonts
}{}
\makeatletter
\@ifundefined{KOMAClassName}{% if non-KOMA class
  \IfFileExists{parskip.sty}{%
    \usepackage{parskip}
  }{% else
    \setlength{\parindent}{0pt}
    \setlength{\parskip}{6pt plus 2pt minus 1pt}}
}{% if KOMA class
  \KOMAoptions{parskip=half}}
\makeatother
\usepackage{xcolor}
\usepackage[margin=1in]{geometry}
\usepackage{longtable,booktabs,array}
\usepackage{calc} % for calculating minipage widths
% Correct order of tables after \paragraph or \subparagraph
\usepackage{etoolbox}
\makeatletter
\patchcmd\longtable{\par}{\if@noskipsec\mbox{}\fi\par}{}{}
\makeatother
% Allow footnotes in longtable head/foot
\IfFileExists{footnotehyper.sty}{\usepackage{footnotehyper}}{\usepackage{footnote}}
\makesavenoteenv{longtable}
\usepackage{graphicx}
\makeatletter
\def\maxwidth{\ifdim\Gin@nat@width>\linewidth\linewidth\else\Gin@nat@width\fi}
\def\maxheight{\ifdim\Gin@nat@height>\textheight\textheight\else\Gin@nat@height\fi}
\makeatother
% Scale images if necessary, so that they will not overflow the page
% margins by default, and it is still possible to overwrite the defaults
% using explicit options in \includegraphics[width, height, ...]{}
\setkeys{Gin}{width=\maxwidth,height=\maxheight,keepaspectratio}
% Set default figure placement to htbp
\makeatletter
\def\fps@figure{htbp}
\makeatother
\setlength{\emergencystretch}{3em} % prevent overfull lines
\providecommand{\tightlist}{%
  \setlength{\itemsep}{0pt}\setlength{\parskip}{0pt}}
\setcounter{secnumdepth}{-\maxdimen} % remove section numbering
\ifLuaTeX
  \usepackage{selnolig}  % disable illegal ligatures
\fi
\IfFileExists{bookmark.sty}{\usepackage{bookmark}}{\usepackage{hyperref}}
\IfFileExists{xurl.sty}{\usepackage{xurl}}{} % add URL line breaks if available
\urlstyle{same}
\hypersetup{
  pdftitle={Informe 1: Encuesta Protección Social de Hogares},
  pdfauthor={Matías Deneken},
  hidelinks,
  pdfcreator={LaTeX via pandoc}}

\title{Informe 1: Encuesta Protección Social de Hogares}
\author{Matías Deneken}
\date{2024-04-29}

\begin{document}
\maketitle

\hypertarget{introducciuxf3n}{%
\subsection{Introducción}\label{introducciuxf3n}}

Para los propositos solicitados, se realizó un adecuación de la Encuesta
Protección Social (EPS) específicamente las olas del 2004, 2006, 2009,
2015 y 2016.

Una primera etapa buscó renombrar las variables de acuerdo a lo
solicitado
\href{https://www.dropbox.com/scl/fi/lujwecrqvs6pza08093ul/Rename-variables.docx?rlkey=mdn8s825i83jrltojfyp7tddf\&dl=0}{\textbf{aquí}}.
Esto con el fin de darle un seguimiento longitudinal a las variables
señaladas en el documento.

Esto se encuentran en diferentes scripts:

\begin{itemize}
\item
  Para la bbdd del 2004 se puede encontrar su procesamiento pinchando
  \href{https://github.com/matdknu/EPS/blob/main/process/base2004.R}{\textbf{aquí}}
\item
  Para la bbdd del 2006 se puede encontrar su procesamiento pinchando
  \href{https://github.com/matdknu/EPS/blob/main/process/base2006.R}{\textbf{aquí}}
\item
  Para la bbdd del 2009 se puede encontrar su procesamiento pinchando
  \href{https://github.com/matdknu/EPS/blob/main/process/base2009.R}{\textbf{aquí}}
\item
  Para la bbdd del 2015 se puede encontrar su procesamiento pinchando
  \href{https://github.com/matdknu/EPS/blob/main/process/base2015.R}{\textbf{aquí}}
\item
  Para la bbdd del 2020 se puede encontrar su procesamiento pinchando
  \href{https://github.com/matdknu/EPS/blob/main/process/base2020.R}{\textbf{aquí}}
\end{itemize}

Una segunda etapa contrastó que tipos de variables se encontraban de
forma similar en cada una de las BBDD.

A partir del procesamiento realizado por mí, y de acuerdo a la
información proporcionada por el procesamiento de la asistente anterior
y mis correciones, esto quedaría así:

\begin{longtable}[]{@{}lllll@{}}
\toprule\noalign{}
Variable & 2004 & 2006 & 2009 & 2015 \\
\midrule\noalign{}
\endhead
\bottomrule\noalign{}
\endlastfoot
folio\_n & ✓ & ✓ & ✓ & ✗ \\
region & ✓ & ✓ & ✓ & ✗ \\
fact\_exp & ✓ & ✓ & ✗ & ✗ \\
parentesco & ✓ & ✓ & ✓ & ✓ \\
sexo & ✓ & ✓ & ✓ & ✗ \\
edad & ✓ & ✓ & ✓ & ✗ \\
ecivil & ✓ & ✓ & ✓ & ✓ \\
asiste & ✓ & ✓ & ✗ & ✓ \\
neduc & ✓ & ✓ & ✗ & ✗ \\
fineduc & ✗ & ✗ & ✗ & ✓ \\
ncarreras & ✓ & ✓ & ✓ & ✓ \\
prueba & ✓ & ✗ & ✗ & ✗ \\
ingotros & ✓ & ✓ & ✗ & ✗ \\
ingnlabe & ✓ & ✓ & ✗ & ✗ \\
vivienda & ✓ & ✓ & ✓ & ✓ \\
montoviv & ✓ & ✓ & ✓ & ✓ \\
ple & ✓ & ✓ & ✗ & ✗ \\
pse & ✓ & ✓ & ✓ & ✓ \\
disc & ✓ & ✓ & ✗ & ✗ \\
disca & ✓ & ✓ & ✗ & ✗ \\
discph & ✗ & ✗ & ✗ & ✓* \\
ncasado & ✓ & ✓ & ✓ & ✓ \\
hogar & ✓ & ✗ & ✓ & ✓ \\
npareja & ✓ & ✓ & ✗ & ✗ \\
edadpareja & ✓ & ✓ & ✓ & ✓ \\
edadconv & ✓ & ✓ & ✓ & ✓ \\
vivepareja & ✓ & ✓ & ✗ & ✗ \\
nhijos & ✓ & ✓ & ✓ & ✓ \\
sexoh & ✓ & ✓ & ✓ & ✓* \\
anoh & ✓ & ✓ & ✓ & ✓* \\
hcasa & ✓ & ✓ & ✓ & ✓* \\
nwork & ✓ & ✓ & ✓ & ✓ \\
anoiw & ✓ & ✓ & ✓ & ✓ \\
anotw & ✓ & ✓ & ✓ & ✓ \\
situationw & ✓ & ✓ & ✓ & ✗ \\
tipow & ✓ & ✓ & ✓ & ✓ \\
catw & ✓ & ✓ & ✓ & ✓ \\
contrato & ✓ & ✓ & ✓ & ✓ \\
inglab & ✓ & ✓ & ✓ & ✓ \\
hworke & ✓ & ✓ & ✓ & ✓ \\
inact & ✓ & ✓ & ✗ & ✗ \\
programa & ✓ & ✓ & ✓ & ✓ \\
npersona & ✓ & ✓ & ✓ & ✓ \\
perteneceh & ✓ & ✓ & ✓ & ✓ \\
edadi & ✓ & ✓ & ✓ & ✓ \\
trabajoi & ✓ & ✓ & ✓ & ✓ \\
contratoi & ✓ & ✗ & ✗ & ✗ \\
inglabi & ✓ & ✓ & ✓ & ✓ \\
hworki & ✓ & ✗ & ✗ & ✗ \\
ingnlabi & ✓ & ✓ & ✓ & ✓ \\
factor\_EPS & ✗ & ✗ & ✗ & ✓ \\
\end{longtable}

\hfill\break

A lo largo de los años estudiados, varias variables han sido
consistentemente usadas, modificadas o eliminadas, reflejando cambios en
la estructuración de datos o en las necesidades de información.

\hypertarget{variables-consistentes}{%
\subsubsection{\texorpdfstring{\textbf{Variables
Consistentes}}{Variables Consistentes}}\label{variables-consistentes}}

Algunas variables han aparecido de manera constante a lo largo de todos
los años analizados, como \textbf{\texttt{parentesco}},
\textbf{\texttt{ecivil}}, \textbf{\texttt{vivienda}}, y
\textbf{\texttt{ncasado}}, lo que indica su importancia sostenida en el
estudio.

\hypertarget{eliminaciones-y-cambios}{%
\subsubsection{\texorpdfstring{\textbf{Eliminaciones y
Cambios}}{Eliminaciones y Cambios}}\label{eliminaciones-y-cambios}}

\begin{itemize}
\item
  \textbf{Eliminaciones Notables}: Variables como
  \textbf{\texttt{fact\_exp}}, \textbf{\texttt{prueba}},
  \textbf{\texttt{ingotros}}, \textbf{\texttt{ple}}, y
  \textbf{\texttt{disc}} fueron eliminadas en años posteriores,
  posiblemente debido a cambios en la metodología de recopilación de
  datos o prioridades de investigación.
\item
  \textbf{Renombramientos}: Se observaron cambios de nombre en algunas
  variables para clarificar su propósito o reflejar mejor la información
  que representan, como \textbf{\texttt{disca}} a
  \textbf{\texttt{discpa}} y \textbf{\texttt{contrato}} a
  \textbf{\texttt{contratow}}.
\end{itemize}

\hypertarget{adiciones-y-desagregaciones}{%
\subsubsection{\texorpdfstring{\textbf{Adiciones y
Desagregaciones}}{Adiciones y Desagregaciones}}\label{adiciones-y-desagregaciones}}

\begin{itemize}
\item
  \textbf{Nuevas Variables en 2015}: Se introdujeron nuevas variables
  como \textbf{\texttt{fineduc}}, \textbf{\texttt{factor\_EPS}}, y
  series desagregadas como \textbf{\texttt{discph1-5}}, indicando un
  enfoque en mayor detalle y especificidad en los datos recogidos.
\item
  \textbf{Detalles Añadidos}: La aparición de múltiples variables
  desagregadas en 2015 como \textbf{\texttt{sexoh1-5}} y
  \textbf{\texttt{anoh1-5}} muestra un esfuerzo por capturar más
  detalles y variabilidad dentro de las categorías existentes.
\end{itemize}

Este análisis revela una evolución en la recopilación y presentación de
datos a lo largo del tiempo, con un esfuerzo por adaptar el conjunto de
datos a nuevas necesidades analíticas y metodológicas, manteniendo
algunas variables fundamentales mientras se introducen nuevas para una
comprensión más profunda de los temas estudiados.

\end{document}
